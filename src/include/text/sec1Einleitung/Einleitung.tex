\section{Einleitung}\label{sec:einleitung}

	\enquote{Einfach einloggen und herunterladen.}
	--- Ein Satz, der im alltäglichen Leben immer mehr an Bekanntheit gewinnt.
	Doch wie funktioniert dieses \enquote{Einloggen und Herunterladen} eigentlich?
	Wie kann gewährleistet werden,
	dass die zu übermittelnden Daten sicher übertragen werden,
	oder die Identität des Benutzers nicht gestohlen wird?
	Wie werden überhaupt Daten übertragen?
	Was sind verteilte Systeme?
	All diese Fragen sollen im Folgenden zumindest ansatzweise beantwortet werden.
	Die Ziele dieser Arbeit werden demnach sein,
	die Datenübertragung über das Internet im Ansatz zu verstehen,
	verschiedene Architekturen zur Realisierung dieser kennenzulernen
	und darüber hinaus die Wichtigkeit und Notwendigkeit der Absicherung dieser aufzufassen.

	Da das Netzwerkprotokoll \gls{www} und damit auch das \gls{http} schon seit 1989 existieren,
	eignet sich eine Literaturarbeit zur Darlegung des Themas am besten.
	Im ersten Kapitel werden anfangs die grundlegenden Funktionsweisen
	und programmiertechnischen Hintergründe definiert
	und erläutert,
	wie ein Datentransfer tatsächlich funktionieren kann.
	Danach werden auf Basis dessen ausgewählte Architekturstile aufgezeigt und verglichen.
	Da für eine sichere Kommunikation eine Verschlüsselung unerlässlich ist,
	folgen Erklärungen und Definitionen über Kryptographie.
	Dieser eher theoretische Einschub führt einerseits zu einem Verfahren,
	welches einen sicheren Informationsaustausch gewährleisten kann,
	und andererseits zur Definition des Begriffes~\enquote{Web-Security}.

	Das Kapitel~\nameref{sec:absicherung-des-informationsaustauschs} behandelt verschiedene \glspl{framework},
	die für die \gls{authentifizierung},
	unter Benutzung der vorhergehenden Protokolle und Grundlagen,
	genutzt werden können.
	Hierbei werden auch die Einsatzgebiete dieser erläutert.
	Die Arbeit wird mit einer Zusammenfassung abgerundet und es wird ein Ausblick darauf gegeben,
	wie diese zu der Entscheidung beigetragen hat,
	welche Verfahren und welche Architekturen in die engere Auswahl
	für die Implementierung eines Web-Services kommen.

	Die Literatur dieser Arbeit besteht ausnahmslos aus Online-Quellen,
	da das Thema mit der Entstehung des Internets verwurzelt ist
	und daher fast ausschließlich alles online zu finden ist.
	Hierbei spielt die Reihe \enquote{Request for Comments} eine große Rolle,
	in welcher viele Protokolle definiert sind.
	Ebenso stellt das \gls{w3c} auch einige Definitionen bereit.
	Nahezu alle Quellen haben gemeinsam,
	dass bei der Erklärung die Sprachkonventionen laut RFC 2119~\cite{rfc2119} genutzt werden,
	was im Folgenden durch eine \textit{kursive} Schrift angezeigt wird.
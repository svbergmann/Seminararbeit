\subsection{OAuth}\label{subsec:oauth}
	Die Bezeichnung OAuth bezieht sich auf zwei unterschiedliche Protokolle,
	welche eine standardisierte,
	sichere \gls{autorisierung} erlauben.
	Die Version 1.0 des Protokolls wurde ab 2006 entwickelt
	und 2010 von der \gls{ietf} veröffentlicht~\cite[Vgl.][]{rfc5849}.
	Im Oktober 2012 wurde die Version 2.0 eingeführt~\cite[Vgl.][]{rfc6749},
	welche heutzutage häufiger in Implementierungen zu finden ist.
	Wichtig dabei ist,
	dass die Versionen nicht kompatibel zueinander sind,
	da sie beispielsweise andere Rollen spezifizieren,
	auf welche im Folgenden näher eingegangen wird.
	Des Weiteren ist das OAuth Protokoll 1.0 unabhängig von dem darunterliegenden Transportprotokoll,
	während sich Version 2.0 vollkommen auf~\nameref{subsec:ssl} stützt.
	Wer Version 1.0 nutzt,
	muss sich also selbst um die Transportsicherheit kümmern,
	kann allerdings jedes Transportprotokoll nutzen,
	während 2.0 nur in Kombination mit \gls{https} genutzt werden kann.
	Dies führt auch zu einem weiteren Aspekt,
	dem der Benutzererfahrung.
	Vor allem bezüglich dieses Unterschieds der Sicherheit ist Version 2.0 leichter zu implementieren
	und benutzerfreundlicher,
	daher wird es auch im Folgenden näher betrachtet.

	\subsubsection{Beispiel}\label{subsubsec:beispiel}
		Zur besseren Übersicht wird der Protokollfluss zuerst anhand eines Beispiels erläutert.
		Es wird angenommen,
		dass eine Applikation namens \enquote{Witz des Tages} existiert,
		welche täglich einen Witz an registrierte E-Mail-Adressen verschickt.
		Ein Benutzer möchte nun seine Bekannten an den Witzen teilhaben lassen
		und die E-Mail-Adressen dieser auch dort eintragen.
		Hierfür möchte er nicht alle Adressen eingeben,
		da diese schon in seinem Kontaktbuch stehen.
		Er hat sich über ein externes Programm an dieser Applikation angemeldet,
		welches im Folgenden als \textbf{Authorization Server} dargestellt ist.

		Der \textbf{Resource Owner} (Benutzer) schickt also eine initiale Anfrage an die \enquote{Witz des Tages} App.
		Daraufhin fragt der \textbf{Client} (Applikation) beim \textbf{Authorization Server}
		zusammen mit einigen Parametern
		(\textbf{Client ID},~\textbf{Redirect \gls{uri}}, \textbf{Response Type} und \textbf{Scope})
		den \textbf{Authorization Code} für die Weitergabe der E-Mail-Adressen an.
		Der \textbf{Resource Owner} (Benutzer) wird jetzt gefragt,
		ob er dem \textbf{Clienten} (Applikation) die angefragte Freigabe erteilen möchte,
		worauf dieser zustimmt.
		Im nächsten Schritt wird der \textbf{Resource Owner} (Benutzer) an den \textbf{Clienten} (Applikation)
		über die \textbf{Redirect \gls{uri}},
		zusammen mit dem generierten \textbf{Authorization Code},
		zurückgeschickt.
		Dieser kann dann damit beim \textbf{Authorization Server},
		in Kombination mit \textbf{Client ID} und \textbf{Client Secret},
		ein \textbf{Access Token} anfragen.
		Hat der \textbf{Client} (Applikation) das \textbf{Access Token} erhalten,
		schickt dieser dies nun an den \textbf{Resource Server}
		und erhält im Austausch dafür die Kontakte.

	\subsubsection{Terminologien}\label{subsubsec:terminologien}
		In OAuth 2.0 existieren also einige Terminologien~\cite[Vgl.][]{rfc6749},
		welche schon in das vorhergehende Beispiel eingearbeitet wurden.
		Diese sind für das Verständnis des Protokolls erforderlich
		und werden im Folgenden aufgeführt und näher erläutert.
		\begin{itemize}
			\item[\textbf{Resource Owner:}] Stellt eine Entität
			(auch \enquote{end-user} genannt, falls es eine Person ist) dar,
			welche Zugriff zu einer geschützten Resource gewährleisten kann.

			\item[\textbf{Client:}] Eine Anwendung,
			welche die geschützte Resource im Namen
			des \textbf{Resource Owners} mit dessen Autorisierung anfragt.
			Dieser Terminus impliziert nicht,
			auf welchem System die Anfrage ausgeführt wird.

			\item[\textbf{Authorization Server:}] Dieser Server \glslink{authentifizierung}{authentifiziert}
			den \textbf{Resource Owner} und stellt dem \textbf{Clienten} Token aus.

			\item[\textbf{Resource Server:}] Ein Server,
			auf welchem diese geschützten Ressourcen liegen.
			Zudem ist dieser dazu in der Lage,
			auf Access Token zu antworten und diese zu akzeptieren.

			\item[\textbf{Redirect \gls{uri}:}] Eine \gls{url},
			auf die der \textbf{Resource Owner} zurückgeschickt wird,
			nachdem dem \textbf{Clienten} die Erlaubnis erteilt wurde.

			\item[\textbf{Response Type:}] Der vom \textbf{Clienten} erwartete Typ der Information.
			Meistens ist dies ein~\textbf{Authorization Code}.

			\item[\textbf{Scope:}] Beschreibt die vom \textbf{Clienten} angefragten Informationen.
			(Bsp.: Die E-Mail-Adressen der Bekannten auslesen und weitergeben.)

			\item[\textbf{Consent:}] Der \textbf{Authorization Server} fragt
			den \textbf{Resource Owner} nach einer Zustimmung,
			ob der \textbf{Client} die angefragten Informationen benutzen darf.
			(Bsp.: \enquote{Die Applikation möchte die E-Mail-Adressen aus dem Kontaktbuch auslesen.
			Zulassen / Verweigern})

			\item[\textbf{Client ID:}] Eine ID für die Identifizierung des
			\textbf{Clienten} beim~\textbf{Authorization Server}.

			\item[\textbf{Client Secret:}] Stellt ein geheimes Passwort dar,
			welches nur \textbf{Client} und \textbf{Authorization Server} kennen.
			Dies wird genutzt, um private Informationen zu teilen.

			\item[\textbf{Authorization Code:}] Ein temporärer Code,
			welcher vom \textbf{Clienten} an den \textbf{Authorization Server} gegeben wird.
			Der \textbf{Client} erhält dafür ein~\textbf{Access Token}.

			\item[\textbf{Access Token:}] Wie der Name schon andeutet,
			beinhaltet dieses Token spezifische Bereiche
			und die Dauer des gewährten Zugangs.
			Um auf eine Resource zuzugreifen,
			muss dieses Token vom Clienten an den \textbf{Resource Server}
			als \gls{autorisierung} übermittelt werden.

			\item[\textbf{Refresh Token:}] Das Refresh Token wird verwendet,
			um beim~\enquote{Authorization Server} ein neues Access Token anzufragen,
			falls das vorherige abgelaufen ist.
			Dadurch kann man die Lebensdauer des Access Tokens kürzer gestalten,
			was die Sicherheit dieses Protokolls erheblich erhöht.
			Ein Angreifer kann sich selbst mit Kenntnis des Access Tokens nur
			in einem gewissen Zeitraum als der eigentliche
			\glslink{autorisierung}{autorisierte} Nutzer ausgeben,
			da das Token seine Gültigkeit verliert.

		\end{itemize}

	\subsubsection{Ablauf der Autorisierung}\label{subsubsec:ablauf-der-autorisierung}
		Der abstrakte Ablauf einer \gls{autorisierung} stellt sich dabei,
		wie in Abbildung~\ref{fig:oauthAbstractFlow} zu sehen ist, dar:
		\begin{enumerate}[(A)]
			\item Der \textbf{Client} beantragt \gls{autorisierung} vom~\textbf{Resource Owner}.
			Diese Anfrage kann entweder direkt an den \textbf{Resource Owner} gestellt werden,
			oder besser indirekt an den~\textbf{Authorization Server}.

			\item Der \textbf{Resource Owner} antwortet mit einer
			\glslink{autorisierung}{Autorisierungsgenehmigung},
			welche eine aus vier zugelassenen Typen sein kann,
			die jeweils abhängig von den unterstützten Typen des \textbf{Authorization Servers} sind.

			\item Der \textbf{Client} fordert ein Access Token beim \textbf{Authorization Server} an
			und schickt die vorher erhaltene \glslink{autorisierung}{Autorisierungsgenehmigung} mit.

			\item Der \textbf{Authorization Server} \glslink{authentisierung}{authentisiert}
			den \textbf{Clienten},
			validiert die \glslink{autorisierung}{Autorisierungsgenehmigung} und
			erstellt ein Access Token.

			\item Der \textbf{Client} fragt die Resource beim \textbf{Resource Server} an
			und \glslink{authentifizierung}{authentifiziert} sich mit dem Access Token.

			\item Der \textbf{Resource Server} prüft das Token und gibt Zugriff auf die Resource.

		\end{enumerate}

		\bigskip\noindent
		Der Ablauf der \gls{autorisierung} ist also meistens nicht genau so,
		wie die Abbildung~\nameref{fig:oauthAbstractFlow} zeigt.
		Bei den meisten Implementierungen interagiert der \textbf{Resource Owner}
		tatsächlich nur zu Anfang mit dem Clienten,
		danach folgt eine Interaktion zwischen \textbf{Authorization Server} und~\textbf{Resource Owner},
		wonach der \textbf{Authorization Code} an den \textbf{Clienten} geschickt wird.
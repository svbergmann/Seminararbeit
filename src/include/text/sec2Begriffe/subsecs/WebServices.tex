\subsection{Web Services}\label{subsec:web-services}
	Die Definition eines Web Services ist laut dem \gls{w3c}
	folgende~\cite{WebServiceW3C}:
	\begin{displayquote}
		A Web service is a software system designed to support interoperable machine-to-machine interaction over a network.
		It has an interface described in a machine-processable format (specifically WSDL).
		Other systems interact with the Web service in a manner prescribed by its description using SOAP messages,
		typically conveyed using HTTP with an XML serialization in conjunction with other Web-related standards.
	\end{displayquote}
	Unter einem Web Service versteht man also ein Softwaresystem,
	welches dafür da ist,
	die Maschine-zu-Maschine Kommunikation über ein Netzwerk zu realisieren.
	Dieses System stellt eine Schnittstelle bereit,
	die in einem maschinenlesbaren Format,
	genauer \gls{wsdl},
	beschrieben ist.
	Andere Systeme können dann über sogenannte \gls{soap} Nachrichten mit dem Service kommunizieren.
	Typischerweise werden diese Botschaften über~\nameref{subsubsec:http} Methoden,
	unter Benutzung einer \gls{xml} Serialisierung,
	in Verbindung mit anderen web-bezogenen Standards,
	übermittelt. \par
	Aus dieser Definition folgt unter anderem,
	dass ein Web Service ein oder mehrere Dienste bereitstellt,
	welche über das Web in Anspruch genommen werden können.
	Jeder Web Service besitzt auch immer einen \gls{uri},
	mit dem dieser eindeutig identifiziert werden kann.
	Wie oben schon genannt,
	erfolgt die Kommunikation mit Web Services ausschließlich zwischen Maschinen,
	womit diese zu Webanwendungen stark abgegrenzt werden,
	bei denen wiederum eine menschliche Interaktion erfolgt.
	Außerdem gibt es zwei Eigenschaften,
	die für die Popularität der Web Services entscheidend sind.

	\paragraph{Plattformunabhängigkeit}
		Client und Server (in diesem Fall der Web Service) können unabhängig
		von ihrer jeweiligen Konfiguration miteinander kommunizieren.
		So kann ein Client, welcher unter Linux läuft,
		beispielsweise problemlos mit einem Windows Server interagieren und umgekehrt.
		Daraus folgt auch,
		dass es eine Schnittstelle zwischen beiden Maschinen geben muss,
		die einheitlich ist.
		Meistens übertragen die Kommunikationspartner die Daten über \gls{xml}, \gls{json} oder ähnliche Strukturen.

	\paragraph{Verteilt}
		Ein Server stellt seinen Dienst nicht nur für genau einen Clienten bereit,
		sondern muss auch mehrere Clienten bedienen können.
\subsection{TLS/SSL}\label{subsec:ssl}
	Bereits neun Monate nach der Verbreitung des ersten Webbrowsers \enquote{Mosaic} 1994
	stellte Netscape Communications die erste Version des \gls{ssl} Protokolls (1.0) vor.
	Nur fünf Monate später kam die Version 2.0,
	zusammen mit dem neuen Browser \enquote{Netscape Navigator},
	auf den Markt.
	Die Firma übergab letztendlich im Sommer 1996 die Versionsverwaltung über die Version 3.0 an die \gls{ietf},
	um einen einheitlichen Internet-Standard zu entwickeln.
	1999 wurde dann \gls{ssl} in \gls{tls} umbenannt und
	unter \enquote{The TLS Protocol Version 1.0}~\cite[Vgl.][]{rfc2246} veröffentlicht.
	Heutzutage werden die Versionen 1.2~\cite[Vgl.][]{rfc5246} und 1.3~\cite[Vgl.][]{rfc8446} genutzt,
	da frühere Standards seit März 2021 unzulässig sind. \par
	Das \gls{tls} Protokoll wird genutzt,
	damit sensible Daten bei der Übertragung über das Internet nicht gelesen oder manipuliert werden können.
	Im Protokollstapel des \glslink{isoOsiModell}{ISO/OSI-Referenzmodells}
	befindet sich \gls{tls} auf der Ebene 5,
	der Sitzungsschicht, auch \enquote{Session-Layer} genannt.
	Diese Schicht ist damit genau zwischen Transport- und Darstellungsschicht,
	also zwischen Protokollen, wie \gls{tcp} und~\nameref{subsubsec:http}.
	Aufgabe hiervon ist es also,
	Daten aus den höheren Ebenen zu verschlüsseln und an die Transportschicht weiterzugeben.
	Genauer besteht das Protokoll aus zwei Schichten,
	auf welche im Folgenden näher eingegangen wird.

	\subsubsection{TLS Handshake Protocol}\label{subsubsec:tlsHandshake}
		Das Handshake Protokoll ermittelt den stärksten,
		gemeinsam unterstützten Algorithmus,
		\glslink{authentifizierung}{authentifiziert} die Kommunikationspartner
		und bestimmt optionaler Weise einen Sitzungsschlüssel für
		die~\hyperref[subsubsec:symVersch]{symmetrische Verschlüsselung}.
		In vereinfachter Darstellung passieren dabei folgende Schritte.
		Der Client schickt eine \enquote{Hello} Nachricht an den Server,
		in welcher dieser alle kryptografischen Informationen,
		wie unterstützte Algorithmen,
		eine Zufallsnummer des Clienten und eine Session ID mitteilt.
		Der Server antwortet seinerseits mit einem \enquote{Hello},
		in welchem der stärkste gemeinsame Algorithmus,
		eine Zufallsnummer des Servers und die Session ID übertragen wird.

	\subsubsection{TLS Record Protocol}\label{subsubsec:tlsRecord}
		Das Record Protocol (engl.\ Record Layer) ist vollständig getrennt vom~\nameref{subsubsec:tlsHandshake}.
		Es verschickt die Daten symmetrisch mit den vorher ausgehandelten Verschlüsselungsalgorithmen und Sitzungsschlüsseln.
		Das Effiziente hierbei ist,
		dass die Sicherheit der~\hyperref[subsubsec:asymVersch]{asymmetrischen Verschlüsselung}
		nur beim Aushandeln des Schlüssels genutzt wird,
		während die eigentlichen Daten dann mit der
		\hyperref[subsubsec:symVersch]{symmetrische Verschlüsselung} verschickt werden.
		Damit geht man dem Problem der Performance-Einbuße aus dem Weg.
\subsection{Netzwerkprotokolle}\label{subsec:netzwerkprotokolle}
	Im Folgenden wird kurz auf relevante Netzwerkprotokolle eingegangen,
	welche sich in der Anwendungsschicht (engl.\ Application Layer) des
	\gls{tcp}/IP-Modells\footnote{s. Abbildung~\ref{fig:schichtenmodell}} befinden.

	\subsubsection{HTTP}\label{subsubsec:http}
		Das \gls{http} existiert bereits seit 1996 in der Version \enquote{\gls{http}/1.0}~\cite[Vgl.][]{rfc1945}
		und definiert ein leichtgewichtiges, schnelles und einfach zu implementierendes Protokoll,
		welches für viele Aufgaben eingesetzt werden kann.
		Es ist zudem zustandslos und objektorientiert, woraus folgt, dass
		jede Operation über eine separate \gls{tcp}-Verbindung realisiert wird.

		1999 wurde \enquote{\gls{http}/1.1}~\cite[Vgl.][]{rfc7230} standardisiert.
		Diese Version stellte eine Verbesserung bezüglich der Ladezeiten dar,
		indem ein Header Feld namens \enquote{Connection: keep-alive} hinzugefügt wurde.
		Dieses Feld kann vom Clienten optional gesetzt werden, um zu signalisieren,
		dass dieser die \gls{tcp}-Verbindung offen halten möchte.
		Des Weiteren wurden \enquote{Non-IP-based Virtual \glspl{host}} durch das Header Feld \enquote{\gls{host}}
		als Lösung für virtuelle \glspl{host} hinzugefügt,
		wodurch sich nun beliebig viele Domänen und Dienste die gleiche IP-Adresse teilen,
		aber trotzdem auf einem Rechner liegen können.

		Eine weitere Beschleunigung der Datenübertragung stellte \enquote{\gls{http}/2.0}~\cite[Vgl.][]{rfc7540} dar.
		Das 2015 standardisierte Protokoll ist abwärtskompatibel zur Version 1.1.
		Es kann aber eine Anfrage vom Clienten über das Header Feld \enquote{Connection: Upgrade} gestartet werden,
		ob der Server die neuere Version unterstützt.
		Falls diese Konstellation gegeben ist, wird nach der Antwort des Servers mit der Version 2.0 weitergearbeitet.
		Da das \gls{pipelining} in Version 1.1 zwar erlaubt ist,
		jedoch von Browser zu Browser entweder unterschiedlich oder gar nicht implementiert ist,
		wurde 2015 das \enquote{Multiplexing} hinzugefügt.
		Die Verbesserung hierbei stellen die Streams dar,
		welche mehrere Anfragen oder Antworten gleichzeitig übertragen können.
		Des Weiteren muss der Server nicht mehr die Reihenfolge der Antworten beachten,
		sondern weist den Streams eine eindeutige ID zu,
		was das sogenannte \gls{headOfLineBlocking} löst.

		Der grundsätzliche Ablauf ist wie folgt:
		\begin{compactenum}
			\item Der Client baut eine \gls{tcp}/IP-Verbindung zum Server auf.
			\item Der Client sendet eine \gls{http}-Anfrage (s.~\nameref{p:httpMethoden}),
			in der das geforderte Dokument / der geforderte Dienst genau spezifiziert ist.
			\item Der Server antwortet mit einem Statuscode und je nach Erfolg der Anfrage auch mit dem angeforderten Dokument.
			\item Im Falle von~\enquote{\gls{http}/1.0} wird die \gls{tcp}/IP-Verbindung geschlossen.
			Wie oben schon erwähnt, kann durch das Setzen des \enquote{Connection: keep-alive} Feldes
			die Verbindung aufrechterhalten werden und so entfallen die Schritte 4 und 1 bei weiteren Anfragen.
		\end{compactenum}

		\paragraph{HTTP-Methoden}\label{p:httpMethoden}
			Eine \gls{http}-Anfrage benutzt Methoden zur Spezifikation der gewünschten Aktion.
			Davon gibt es seit \enquote{\gls{http}/1.1} folgende acht Methoden,
			von welchen alle Server die Methoden \textbf{GET} und \textbf{HEAD} unterstützen~\textit{müssen}.
			Alle anderen Methoden sind optional.
			\begin{compactitem}
				\item \textbf{GET} fordert eine im Request-\glsname{uri} definierte Resource an.
				\item \textbf{HEAD} ist ähnlich wie GET, übermittelt aber nur den Header ohne den Body.
				\item \textbf{POST} überträgt Daten an den Server.
				\item \textbf{PUT} möchte alle Daten, welche momentan auf dem Server liegen,
				mit den eigenen Daten überschreiben.
				\item \textbf{DELETE} löscht Dokumente auf dem Server (sofern die Rechte dazu vorhanden sind).
				\item \textbf{CONNECT} kann einen Tunnel zum Server für die bidirektionale Kommunikation öffnen.
				\item \textbf{OPTIONS} fordert den Zugriffspfad einer Resource an und die weitere Kommunikation,
				welche darüber möglich ist.
				\item \textbf{TRACE} ist zum Testen.
				Die Anfrage wird vom Server zurückgeschickt.
			\end{compactitem}

	\subsubsection{FTP}\label{subsubsec:ftp}
		Im Oktober 1985 wurde das \gls{ftp} spezifiziert~\cite[Vgl.][]{rfc959}.
		Seitdem ist dies der Internet-Standard für die Übertragung von Dateien und wird benutzt,
		um eine komplette Datei von einem auf den anderen Rechner zu kopieren
		oder interaktiven Zugriff zu ermöglichen.
		Hierbei werden zwei Ports genutzt,
		wobei Port 20 und Port 21 der Standard sind.
		Über den Control Port des Servers (Port 21) wird eine \gls{tcp}-Verbindung aufgebaut,
		mit der die Befehle gesendet werden.
		Falls dies erfolgreich verlaufen ist,
		wird über einen anderen Port (standardmäßig Port 20) die Dateiübertragung abgewickelt.
		Normalerweise ist die Vorgehensweise etwas komplizierter,
		funktioniert je nach aktivem oder passivem Modus etwas anders und besitzt seinerseits auch Methoden.
		Das ist hier aber nicht von Belang und würde den Rahmen der Arbeit sprengen.
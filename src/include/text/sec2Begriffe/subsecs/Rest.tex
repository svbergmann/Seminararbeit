\subsection{REST}\label{subsec:rest}
	Anders als \gls{soap} ist die \gls{rest} \gls{api} kein offizielles Protokoll,
	sondern beschreibt eine Schnittstelle,
	welche sich an den Prinzipien und Denkweisen des \gls{www},
	demzufolge auch an den~\nameref{p:httpMethoden},
	orientiert.
	Das bedeutet unter anderem,
	dass keine strikten Regeln festgelegt,
	sondern eher Empfehlungen zur Implementierung ausgegeben werden.
	Somit liegt der Schwerpunkt einer \glslink{rest}{RESTful} \gls{api} darin,
	die Maschine-zu-Maschine Kommunikation über mehrere unterschiedliche Programmiersprachen zu ermöglichen.
	Dabei hat \gls{rest} sechs Architekturprinzipien, welche im Folgenden dargestellt werden.

	\subsubsection{Client-Server}
		Das Client-Server-Modell, auch Client-Server-Prinzip genannt, fordert eine strikte Aufgabenteilung.
		Der Server stellt Dienste bereit, die von (meistens mehreren) Clienten,
		unter Umständen auch gleichzeitig, genutzt werden können.
		Dadurch entsteht eine Architektur,
		in der die Server passiv agieren.
		Diese warten auf eingehende Verbindungswünsche und Anfragen
		und die Clienten fordern aktiv einen Dienst an.
		Ein Dienst definiert in diesem Fall wie
		und welche Daten ausgetauscht werden sollen.

	\subsubsection{Zustandslosigkeit}
		Clienten und Server müssen zustandslos kommunizieren, was bedeutet,
		dass es (streng genommen) keine \enquote{Sitzungsverwaltung} gibt.
		Dadurch folgt, dass jede \gls{rest} Nachricht immer alle Informationen bezüglich des Clienten und Server beinhalten muss.
		Dies begünstigt dann unter anderem die \enquote{Skalierbarkeit} eines Web Services,
		sprich die Aufgabenverteilung auf beliebig viele Maschinen wird erleichtert.

	\subsubsection{Caching}
		Das Caching wird genutzt, um die Effizienz und Schnelligkeit der Anwendung zu verbessern.
		Die Informationen müssen dann entsprechend als \enquote{cacheable} oder eben \enquote{non-cacheable} gekennzeichnet werden.
		Wenn also eine Nachricht \enquote{cacheable} beinhaltet,
		so werden die Nutzdaten clientseitig im \enquote{Cache} gespeichert und bei jeder erneuten Anfrage an den Server abgerufen.
		Dies birgt natürlich das Risiko,
		veraltete Daten aus dem Cache zu bekommen,
		weshalb die Kennzeichnung so wichtig ist.

	\subsubsection{Einheitliche Schnittstelle}
		\gls{rest} setzt auf eine einheitliche Schnittstelle,
		damit alle Dienste gleich angesprochen werden können.
		Dieses Prinzip unterscheidet sich nochmals in vier Aspekte.

		\paragraph{Adressierbarkeit von Ressourcen}
			Jede \glslink{rest}{RESTful} \gls{api} hat eine eindeutige Adresse, den \gls{url},
			welche nur durch die Methoden des Netzwerk Protokolls~\nameref{subsubsec:http} manipuliert werden können.
			Abbildung~\ref{fig:restKommunikation} zeigt ein Beispiel für die Adressierung von Objekten.
			Zu sehen ist,
			dass die Anfrage direkt auf eine Methode des Web Services routet,
			welche den Request dann bearbeiten und die Antwort zurückschicken kann.

		\paragraph{Repräsentationen zur Veränderung von Ressourcen}
			Jede so angefragte Ressource kann die Antwort in unterschiedlichen Darstellungsformen,
			beispielsweise in \gls{html}, \gls{json} oder \gls{xml} kodieren.
			Alle unter dieser Adresse zugänglichen Dienste können dabei voneinander verschiedene Repräsentationen haben,
			die aber dringend für eine \glslink{rest}{RESTful} \gls{api} gleich sein sollten.

		\paragraph{Selbstbeschreibende Nachrichten}
			Nachrichten, welche über den \gls{rest} Dienst verschickt werden, sollen selbst beschreibend sein.
			Jede Anfrage soll also genau das darstellen,
			wofür die deklarierte Methode gedacht ist.
			Eine \enquote{DELETE} Nachricht soll zum Beispiel nur das Löschen anfordern und keine anderen Anfragen.

		\paragraph{HATEOAS}\label{p:resthateoas}
			Das Prinzip der \gls{hateoas} ist laut
			Roy Fielding\footnote{Roy Thomas Fielding ist ein US-amerikanischer Informatiker,
	welcher maßgeblich zur Entwicklung des \gls{http} und der \gls{rest} Architektur beitrug.}
			die wichtigste Eigenschaft der \gls{rest} Architektur.
			Wichtig ist hierbei,
			dass der Client als Antwort vom Server immer alle aktuell verfügbaren \glspl{url} bekommt
			und damit mit den sogenannten \enquote{\glspl{followUpLink}} weiter mit dem Server kommunizieren kann.

	\subsubsection{Mehrschichtige Systeme}
		Ganz nach dem Vorbild des Schichtensystems des Internets\footnote{s. Abbildung~\ref{fig:schichtenmodell}}
		setzt auch \gls{rest} auf die Mehrschichtigkeit der Systeme.
		Somit kann ein Client die oberste Schicht sehen und ansprechen,
		was aber intern eigentlich passiert,
		bleibt dem Server vorbehalten und verborgen.
		Dadurch kann eine bessere Übersichtlichkeit und Skalierbarkeit erreicht werden.
		Der Nachteil daran ist natürlich der \gls{overhead} durch zusätzliche Funktionsaufrufe.

	\subsubsection{Code on Demand (optional)}
		Diese letzte und optionale Bedingung beschreibt,
		dass der Client, zur besseren Bedienung des Servers,
		Code, zum Beispiel in Form von Skripten oder ähnlichem,
		nachladen und clientseitig ausführen kann.
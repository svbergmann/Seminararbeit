% Akronyme ohne Erklärung

\newacronym{aes}{AES}{Advanced Encryption Standard}
\newacronym{des}{DES}{Data Encryption Standard}
\newacronym{flip}{FLIP}{Flight Information Publication}
\newacronym{ftp}{FTP}{File Transfer Protocol}
\newacronym{hateoas}{HATEOAS}{Hypermedia as the Engine of Application State}
\newacronym{http}{HTTP}{Hypertext Transfer Protocol}
\newacronym{https}{HTTPS}{Hypertext Transfer Protocol Secure}
\newacronym{lfid}{LFID}{Luftfahrt Informations Datenbank}
\newacronym{rest}{REST}{Representational State Transfer}
\newacronym{soap}{SOAP}{Simple Object Access Protocol}
\newacronym{ssl}{SSL}{Secure Sockets Layer}
\newacronym{tcp}{TCP}{Transmission Control Protocol}
\newacronym{tls}{TLS}{Transport Layer Security}
\newacronym{smtp}{SMTP}{Simple Mail Transfer Protocol}

% Akronyme mit Erklärung

\newdefineabbreviation
{api}{API}{Application Programming Interface}
{Satz von Befehlen, Protokollen und Funktionen,
	welche Entwickler für die Erstellung von Software zur Ausführung
	allgemeiner Operationen verwenden können.
	Diese Programmierschnittstelle stellt somit die Grundlage
	für die Kommunikation zwischen Anwendungen bereit.}

\newdefineabbreviation
{corba}{CORBA}{Common Object Request Broker Architecture}
{Objektorientiertes Kommunikationsprotokoll,
	welches plattformübergreifend und unabhängig von Programmiersprachen ist}

\newdefineabbreviation
{dcom}{DCOM}{Distributed Component Object Model}
{Von Microsoft entwickeltes Protokoll um auf Basis des Component Object Models (COM)
	über das Netzwerk kommunizieren zu können}

\newdefineabbreviation
{xml}{XML}{Extensible Markup Language}
{Auszeichnungssprache zur Darstellung von Objekten oder Dateien in einem menschen-
und maschinenlesbaren Format}

\newdefineabbreviation
{html}{HTML}{Hypertext Markup Language}
{Sprache zur Darstellung von Inhalten über einen Web Browser}

\newdefineabbreviation
{json}{JSON}{JavaScript Object Notation}
{Maschinenlesbare Sprache zur Darstellung von Objekten.
Bei \gls{javascript} kann dies unter anderem für eine
Instanziierung neuer Objekte genutzt werden.}

\newdefineabbreviation
{uri}{URI}{Uniform Resource Identifier}
{Eindeutige Adressierung von abstrakten und
physikalischen Ressourcen im Internet\\
\glsname{uri}: \glsname{url}s $\cup$ \glsname{urn}s}

\newdefineabbreviation
{url}{URL}{Uniform Resource Locator}
{Adressierung von Informationsobjekten mit Festlegung
des Zugangs-Protokolls (Ort der Ressource)}

\newdefineabbreviation
{urn}{URN}{Uniform Resource Name}
{Adressierung von Objekten ohne ein Protokoll festzulegen
	(Eindeutige und gleichbleibende Referenz - Name der Ressource)}

\newdefineabbreviation
{w3c}{W3C}{World Wide Web Consortium}
{Konsortium, welches aus verschiedenen Mitgliedsorganisationen besteht,
	deren Ziel es ist, Protokolle und Richtlinien des \glsname{www} zu standardisieren}

\newdefineabbreviation
{wsdl}{WSDL}{Web Services Description Language}
{Auszeichnungssprache um Web Services zu charakterisieren}

\newdefineabbreviation
{ws-addressing}{WS-Addressing}{Web Services Addressing}
{Mechanismus für Web Services um Adressinformationen austauschen zu können}

\newdefineabbreviation
{ws-reliableMessaging}{WS-ReliableMessaging}{Web Services Reliable Messaging}
{Standard, welcher garantiert, dass Nachrichten auf jeden Fall beim Empfänger ankommen}

\newdefineabbreviation
{ws-security}{WS-Security}{Web Services Security}
{Kommunikationsprotokoll, welches ermöglicht,
	Sicherheitsaspekte bei Web Services einzubeziehen}

\newdefineabbreviation
{www}{WWW}{World Wide Web}
{Dezentrales Kommunikationsnetz, welches dazu dient,
	Informationen zu übertragen und nutzbar zu machen}

\newdefineabbreviation
{wpa2}{WPA2}{Wi-Fi Protected Access 2}
{Standard für die \gls{authentifizierung} und Verschlüsselung von WLANs,
	basierend auf dem IEEE 802.11 Standard}

\newdefineabbreviation
{ssh}{SSH}{Secure Shell}
{Netzwerkprotokoll, welches dazu befähigt, über ein ungesichertes Netzwerk,
	sicher auf einen Computer zuzugreifen}

\newdefineabbreviation
{7zip}{7z}{7-Zip}
{Freies Dateiformat mit der Endung .7z,
	was zur Komprimierung und Archivierung von Daten verwendet wird}

\newdefineabbreviation
{ietf}{IETF}{Internet Engineering Task Force}
{Organisation zur Verbesserung und Weiterentwicklung der Funktionsweise des Internets}

% Glossareinträge ohne Akronym

\newglossaryentry{pipelining}{
	name=Pipelining,
	description={Falls ein Client persistente Verbindungen unterstützt,
	können die Anfragen hintereinander geschickt werden,
	ohne auf die jeweilige Antwort zu warten.
	Der Server muss dann in genau dieser Reihenfolge auf die Anfragen antworten.}
}

\newglossaryentry{headOfLineBlocking}{
	name=Head-of-Line-Blocking,
	description={Problem, das entsteht,
	wenn mehrere zu versendende Pakete durch das erste Paket aufgehalten werden}
}

\newglossaryentry{followUpLink}{
	name=Follow-up-Link,
	description={Link, der nach einer Aktion auf dem Server an den Clienten zurückgeschickt wird,
	um verfügbare Aktionen/Ressourcen zu signalisieren}
}

\newglossaryentry{host}{
	name=Host,
	description={Computer,
	welcher eine eindeutige IP-Adresse besitzt und darüber erreicht werden kann}
}

\newglossaryentry{eulerphi}{
	name=Eulersche $\phi$-Funktion,
	description={Gibt für jedes $n\in\mathbb{N}$ an, wie viele zu $n$ teilerfremde
	$m\in\mathbb{N}$ mit $m<n$ existieren}
}

\newglossaryentry{isoOsiModell}{
	name=ISO/OSI-Referenzmodell,
	description={Modell für die Netzwerkprotokolle als Darstellung
	über eine Schichtenarchitektur,
	siehe auch~\nameref{fig:schichtenmodell}}
}

\newglossaryentry{overhead}{
	name=Overhead,
	description={Daten, welche nicht primär zu den Nutzdaten zählen,
	aber trotzdem zusätzlich als Information beispielsweise zur Weiterverarbeitung oder
	Speicherung benötigt werden}
}

\newglossaryentry{framework}{
	name=Framework,
	description={Zu deutsch \enquote{Rahmenwerk},
	was eine wiederverwendbare Struktur bereitstellt,
	welche für die Entwicklung verschiedener Programme genutzt werden kann}
}

\newglossaryentry{base64}{
	name=Base64,
	description={Kodierungsverfahren,
	welches den Text zuerst in einen Bytestrom umwandelt,
	aus diesem jeweils immer drei Bytes hernimmt,
	die 24 Bits wiederum in 4 Blöcke à 6 Bit aufteilt und
	letztendlich die Binärwerte wieder in ein Zeichen umwandelt.
	So entsteht bei $n$ zu kodierenden Zeichen ein Bedarf von
	$z=4\cdot \lceil \frac{n}{3} \rceil$ Zeichen.}
}

\newglossaryentry{string}{
	name=String,
	description={Zeichenkette,
	welche in manchen Programmiersprachen einen eigenen Datentypen darstellt}
}

\newglossaryentry{hashfunktion}{
	name=Hashfunktion,
	description={Surjektive Abbildung $h:K\to S$ mit $\vert K\vert \geq \vert S\vert$,
	welche eine Basismenge auf eine kleinere Zielmenge abbildet}
}

\newglossaryentry{singleSignOn}{
	name=Single Sign-On,
	description={Auch als \enquote{Einmalanmeldung} bekannt,
	berechtigt den Clienten sich nur einmal zu \glslink{authentifizierung}{authentifizieren} und
	danach alle bereitgestellten Dienste ohne erneute Anmeldung zu nutzen.}
}

\newglossaryentry{authentisierung}{
	name=Authentisierung,
	description={Ein Nutzer legt Nachweise vor,
	welche dessen Identität bestätigen sollen
	(Behauptung einer Identität)}
}

\newglossaryentry{authentifizierung}{
	name=Authentifizierung,
	description={Stellt die Prüfung der behaupteten \gls{authentisierung} dar
	(Verifizierung der Identität)}
}

\newglossaryentry{autorisierung}{
	name=Autorisierung,
	description={Nach erfolgreicher \gls{authentifizierung} werden spezielle Rechte
	an den Nutzer vergeben
	(Vergabe oder Verweigerung von Rechten)}
}

\newglossaryentry{javascript}{
	name=JavaScript,
	description={Skriptsprache, welche ursprünglich für das dynamische Erstellen von Internetseiten entwickelt wurde}
}